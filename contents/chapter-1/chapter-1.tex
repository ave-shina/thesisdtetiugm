\chapter{Pendahuluan}

\section{Latar Belakang}

Fakultas Teknik Universitas Gadjah Mada merupakan salah satu fakultas yang menyelenggarakan pendidikan dan penelitian di bidang Teknik yang terbagi menjadi beberapa program studi di setiap departemen dengan berbagai fasilitas penunjang kegiatan akademik dan penelitian di dalamnya, Beberapa fasilitas yang dimiliki antara lain ruang kelas, laboratorium, perpustakaan, kantin, lahan parkir dan fasilitas lainnya.

Namun, tidak semua orang dapat mengunjungi fakultas ini secara langsung untuk mengetahui informasi dan tata letak gedung beserta fasilitas yang dimiliki. Berangkat dari permasalahan tersebut, Fakultas Teknik Universitas Gadjah Mada telah membuat peta dua dimensi yang dibagikan melalui website resmi mereka agar dapat dengan mudah diakses melalui internet oleh masyarakat luas, Peta yang seharusnya menjadi gambaran representatif atau ilustrasi permukaan daerah tertentu yang menunjukkan informasi posisi, batas wilayah, dan objek penting yang terdapat di dalamnya serta dengan fungsi dari peta yaitu sebagai alat media informasi, navigasi, perencanaan, analisis, dan visualisasi. Sayangnya, pada peta tersebut hanya terbatas pada informasi titik lokasi dari setiap departemen yang terdapat di sana dan belum memuat informasi mengenai gedung-gedung dan fasilitas-fasilitas penunjang lain yang terdapat pada fakultas tersebut.

Ketidaklengkapannya informasi pada peta fakultas teknik dapat menimbulkan beberapa dampak kerugian seperti kesulitan dalam menemukan lokasi ruangan atau fasilitas yang dibutuhkan, terutama bagi orang yang baru pertama kali mengunjungi fakultas tersebut, kemudian terdapat juga potensi terjadinya kesalahan atau ketidakakuratan dalam mengambil keputusan yang berkaitan dengan penggunaan fasilitas atau ruangan yang tidak tercantum pada peta, keterbatasan akses informasi bagi orang yang memiliki kebutuhan khusus atau disabilitas, seperti informasi mengenai aksesibilitas ruangan atau fasilitas yang ramah disabilitas, berpotensi menimbulkan ketidakpuasan atau kekecewaan bagi pengguna fakultas yang mengharapkan informasi yang lebih lengkap dan potensi menghambat produktivitas dan efisiensi dalam kegiatan yang dilakukan di fakultas teknik jika terdapat kekurangan informasi yang penting pada peta.

Seiring dengan perkembangan teknologi informasi, penggunaan media digital semakin meningkat dalam kehidupan sehari-hari. Salah satu bentuk media digital yang sedang berkembang saat ini adalah teknologi website yang merupakan sebuah kumpulan halaman digital yang terhubung dan dapat diakses melalui internet. Halaman-halaman ini dapat berisi teks, gambar, audio, video, dan berbagai jenis konten lainnya yang dapat diakses oleh pengguna melalui browser di suatu perangkat. Teknologi Website dapat menghadirkan pengalaman interaktif yang dapat digabungkan menggunakan teknologi 3D yang mendalam dan realistis karena mampu mereplikasi objek dan lingkungan dalam tiga dimensi, dengan detail yang mendekati dengan objek dunia nyata. Teknologi 3D tersebut dapat menghasilkan efek bayangan, cahaya, tekstur, dan refleksi yang realistis, serta memungkinkan pengguna untuk berinteraksi dengan objek dalam lingkungan virtual yang dibuat dengan lebih alami. Selain itu juga dapat digabungkan dengan foto panorama 360 yang dapat memperlihatkan pemandangan dari gambar yang memanjang secara horizontal hingga membentuk sudut 360 derajat, sehingga memberikan tampilan yang menyerupai lingkungan 3D seolah-olah kita berada di tengah-tengah lingkungan tersebut dan dapat melihat ke segala arah.

Oleh karena itu, Penelitian tentang Pengembangan 3D Virtual Map Tour Fakultas Teknik Universitas Gadjah Mada Berbasis Web ini diharaplan dapat membantu masyarakat umum untuk dapat mengakses informasi dengan lebih interaktif dan informatif dalam mengenal Fakultas Teknik Universitas Gadjah Mada. Penelitian ini juga diharapkan dapat memberikan kontribusi pada pengembangan teknologi dan aplikasi serupa di berbagai sektor lainnya, seperti pariwisata, pemerintahan dan industri properti sebagai media informasi dan promosi.
% %HAPUS YANG TIDAK PERLU
% %-------------------------------------------------
% \noindent\textbf{Contoh latar belakang penelitian untuk teknik elektro:} \\
% \noindent\fbox{%
% 	\parbox{\textwidth}{%
				
% 		\hspace{1cm} "Peningkatan konsumsi energi listrik yang terus menerus menyebabkan ketersediaan 
% 		sumber energi yang semakin terbatas. Sumber energi terbarukan seperti solar dan angin 
% 		menjadi solusi yang menjanjikan untuk memenuhi kebutuhan energi. Namun, kapasitas 
% 		produksi dan efisiensi dari sumber energi terbarukan masih sangat tergantung pada 
% 		kondisi cuaca dan geografis. Oleh karena itu, perlu adanya sistem penyimpanan energi 
% 		yang efisien dan dapat memastikan ketersediaan energi listrik secara kontinu. Penelitian 
% 		ini akan mengevaluasi kemampuan superkapasitor dalam menyimpan dan mengirimkan 
% 		energi secara efisien dan memastikan ketersediaan energi listrik secara kontinu."
		
% 	}%
% }

% %-------------------------------------------------	
% \vspace{5mm}
% Latar belakang ini memperkenalkan masalah ketersediaan sumber energi dan 
% peningkatan konsumsi energi listrik yang terus menerus. Ini juga memperkenalkan solusi 
% yang menjanjikan dari sumber energi terbarukan dan menjelaskan mengapa perlu adanya 
% sistem penyimpanan energi yang efisien. Latar belakang ini memberikan dasar yang kuat 
% bagi perumusan masalah dan tujuan penelitian, memastikan bahwa hasil penelitian 
% memiliki relevansi dan signifikansi bagi bidang terkait.

%HAPUS YANG TIDAK PERLU
%-------------------------------------------------
% \noindent\textbf{Contoh latar belakang penelitian untuk teknik biomedis:} \\
% \noindent\fbox{%
% 	\parbox{\textwidth}{%
		
% 		\hspace{1cm} "Diagnosis dan pengobatan penyakit memerlukan integrasi informasi medis yang 
% 		akurat dan terkini. Alat diagnostik tradisional seperti CT scan dan MRI memiliki resolusi 
% 		yang tinggi dan dapat mengidentifikasi masalah pada tingkat sel, tetapi sering 
% 		memerlukan banyak waktu dan biaya. Alat deteksi dini seperti tes darah dan urin memiliki 
% 		biaya rendah dan mudah digunakan, tetapi sering kurang akurat dan tidak memberikan gambaran yang jelas tentang masalah medis. Oleh karena itu, penting untuk menemukan metode baru yang memadukan keunggulan dari kedua jenis alat tersebut. Penelitian ini akan mengevaluasi kemampuan nanopartikel dalam meningkatkan akurasi dan efisiensi diagnosis medis."
		
% 	}%
% }

% %-------------------------------------------------	
% \vspace{5mm}
% Latar belakang ini memperkenalkan masalah diagnostik dan pengobatan penyakit dan mempertimbangkan pentingnya integrasi informasi medis yang akurat dan terkini. Ini 
% juga memperkenalkan kelemahan dari alat diagnostik tradisional dan deteksi dini dan menjelaskan mengapa penting untuk menemukan metode baru yang memadukan 
% keunggulan dari kedua jenis alat tersebut. Latar belakang ini memberikan dasar yang kuat bagi perumusan masalah dan tujuan penelitian, memastikan bahwa hasil penelitian 
% memiliki relevansi dan signifikansi bagi bidang terkait.

%HAPUS YANG TIDAK PERLU
%-------------------------------------------------
% \noindent\textbf{Contoh latar belakang penelitian untuk teknologi informasi:} \\
% \noindent\fbox{%
% 	\parbox{\textwidth}{%
		
% \hspace{1cm} "Dengan perkembangan teknologi informasi yang sangat pesat dalam beberapa tahun terakhir, penyimpanan data menjadi masalah yang semakin penting. Semakin banyak data yang diterima setiap hari, semakin penting bagi organisasi untuk memastikan bahwa data 
% mereka aman dan terlindungi. Pada saat yang sama, organisasi juga membutuhkan akses cepat dan efisien ke data mereka untuk membuat keputusan yang tepat. Teknologi 
% enkripsi kuantum baru-baru ini muncul sebagai solusi potensial untuk memenuhi kebutuhan ini, dengan menawarkan tingkat keamanan yang jauh lebih tinggi dan proses 
% enkripsi yang lebih cepat dibandingkan dengan teknologi enkripsi konvensional. Oleh karena itu, penting untuk mengevaluasi efektivitas dan keamanan teknologi enkripsi 
% kuantum dalam sistem penyimpanan data cloud."
		
% 	}%
% }

% %-------------------------------------------------	
% \vspace{5mm}
% Latar belakang ini memperkenalkan masalah penyimpanan data dan mempertimbangkan pentingnya keamanan data. Ini juga memperkenalkan teknologi enkripsi kuantum sebagai solusi potensial dan menjelaskan mengapa evaluasi teknologi ini penting bagi bidang teknologi informasi. Latar belakang ini memberikan dasar yang kuat bagi perumusan masalah dan tujuan penelitian, memastikan bahwa hasil penelitian memiliki relevansi dan signifikansi bagi bidang terkait.

\section{Rumusan Masalah}

Dari latar belakang diatas dapat dirumuskan beberapa permasalahan, antara lain :

\begin{enumerate}
	\item Bagaimana cara merancang dan mengembangkan 3D Virtual Map Tour dalam mengakses informasi dan navigasi gedung-gedung Fakultas Teknik Universitas Gadjah Mada yang dapat diakses melalui web  ?
	
	\item Apa saja fitur-fitur yang dibutuhkan dalam pengembangan 3D Virtual Map Tour Fakultas Teknik Universitas Gadjah Mada untuk meningkatkan pengalaman pengguna?

	\item Bagaimana performa web 3D Virtual Map Tour Fakultas Teknik Universitas Gadjah Mada dalam hal kecepatan, akurasi, dan ketersediaan informasi?
	

\end{enumerate}
	

% %-------------------------------------------------
% \noindent\fbox{%
% 	\parbox{\textwidth}{%
% \noindent\textbf{Contoh} perumusan masalah untuk \textbf{Teknik Elektro}: \\		

% \hspace{1cm} \textbf{"Bagaimana memperbaiki efisiensi penghematan energi pada sistem pencahayaan rumah tangga melalui implementasi teknologi kontrol otomatis?"} \\
	
% \hspace{1cm} Perumusan masalah ini jelas dan spesifik dan menentukan fokus penelitian pada perbaikan efisiensi penghematan energi dalam sistem pencahayaan rumah tangga dengan menggunakan teknologi kontrol otomatis. Ini juga mempertimbangkan latar belakang tentang pentingnya penghematan energi dan memberikan solusi praktis melalui implementasi teknologi. Perumusan masalah ini memberikan dasar yang kuat untuk tujuan dan hipotesis penelitian dan memastikan bahwa hasil penelitian akan berguna bagi bidang teknik elektro.
		
% 	}%
% }

%-------------------------------------------------
% \noindent\fbox{%
% 	\parbox{\textwidth}{%
% \noindent\textbf{Contoh} perumusan masalah untuk \textbf{Teknik Biomedis}: \\		

% \hspace{1cm} \textbf{"Bagaimana memperbaiki akurasi deteksi kanker payudara dengan menggunakan teknologi pemindaian ultrasonografi berbasis AI?"} \\

% \hspace{1cm} Perumusan masalah ini jelas dan spesifik dan menentukan fokus penelitian pada 
% perbaikan akurasi deteksi kanker payudara dengan menggunakan teknologi pemindaian ultrasonografi berbasis AI. Ini mempertimbangkan latar belakang tentang pentingnya deteksi dini kanker payudara dan memberikan solusi praktis melalui implementasi teknologi. Perumusan masalah ini memberikan dasar yang kuat untuk tujuan dan hipotesis penelitian dan memastikan bahwa hasil penelitian akan berguna bagi bidang teknik biomedis.
		
% 	}%
% }

%-------------------------------------------------	
% \noindent\fbox{%
% 	\parbox{\textwidth}{%
% \noindent\textbf{Contoh} perumusan masalah untuk \textbf{Teknologi Informasi}: \\		

% \hspace{1cm} \textbf{"Bagaimana meningkatkan efisiensi dan keamanan sistem penyimpanan data cloud melalui implementasi teknologi enkripsi kuantum?"} \\

% \hspace{1cm} Perumusan masalah ini jelas dan spesifik dan menentukan fokus penelitian pada peningkatan efisiensi dan keamanan sistem penyimpanan data cloud dengan menggunakan teknologi enkripsi kuantum. Ini mempertimbangkan latar belakang tentang pentingnya keamanan data dan memberikan solusi praktis melalui implementasi teknologi. Perumusan masalah ini memberikan dasar yang kuat untuk tujuan dan hipotesis penelitian dan memastikan bahwa hasil penelitian akan berguna bagi bidang teknologi informasi.
		
% 	}%
% }

%-------------------------------------------------	

\section{Tujuan Penelitian}

Tujuan dari penelitian tugas akhir ini yaitu:

\begin{itemize}
	\item Akan dikembangkan suatu aplikasi berbasis website yang dapat diakses browser pada perangkat yang terhubung dengan internet. 
	\item Akan dikembangkan suatu aplikasi berbasis website yang dapat diakses browser pada perangkat yang terhubung dengan internet.
	\item Menampilkan rute jalan yang dapat dilalui oleh kendaran dan jalan kaki dengan disertai tampilan foto panorama 360. 
	\item Setiap lokasi akan dikelompokkan dengan kategori untuk memudahkan dalam pencarian.
	\item Tersedia 2 pilihan bahasa yang dapat dipilih yaitu Bahasa Indonesia dan Bahasa Inggris.
	\item Tersedia 2 pilihan bahasa yang dapat dipilih yaitu Bahasa Indonesia dan Bahasa Inggris.
	\item Terdapat 2 pilihan tema tampilan yang dapat dipilih yaitu gelap dan terang.
\end{itemize}


% \newpage
% \vspace{5mm}
% \textbf{Contoh Tujuan Penelitian Skripsi Teknik Elektro:}

% \begin{minipage}{0.92\textwidth}
% Berikut adalah beberapa contoh tujuan penelitian yang sesuai dengan topik 
% “perbaikan efisiensi penghematan energi pada sistem pencahayaan rumah tangga 
% melalui implementasi teknologi kontrol otomatis”:
% \end{minipage}

% %-------------------------------------------------	
% \noindent\fbox{%
% 	\parbox{\textwidth}{%
% \begin{enumerate}
% \item Menganalisis tingkat efisiensi energi pada sistem pencahayaan rumah tangga 
% sebelum dan setelah implementasi teknologi kontrol otomatis.
% \item Mengukur pengurangan biaya listrik setelah implementasi teknologi kontrol 
% otomatis pada sistem pencahayaan rumah tangga.
% \item Menunjukkan bagaimana teknologi kontrol otomatis dapat memperbaiki 
% efisiensi penghematan energi pada sistem pencahayaan rumah tangga.
% \item Meningkatkan kenyamanan dan keamanan pengguna rumah tangga melalui 
% penggunaan teknologi kontrol otomatis pada sistem pencahayaan.
% \item Menjelaskan bagaimana implementasi teknologi kontrol otomatis 
% mempengaruhi kualitas cahaya dan faktor-faktor lain dalam sistem pencahayaan 
% rumah tangga.
% \item Membandingkan efisiensi energi dan biaya pada sistem pencahayaan rumah 
% tangga dengan teknologi kontrol otomatis dan sistem manual.
% \item Menunjukkan implikasi dan rekomendasi dari hasil penelitian bagi rumah tangga 
% dan lingkungan.
% \end{enumerate}
		
% 	}%
% }

%-------------------------------------------------	

% \newpage
% \vspace{5mm}
% \textbf{Contoh Tujuan Penelitian Skripsi Teknik Biomedis:}

% \begin{minipage}{0.92\textwidth}
% Berikut adalah beberapa contoh tujuan penelitian untuk penelitian dengan tema "Bagaimana memperbaiki akurasi deteksi kanker payudara dengan menggunakan 
% teknologi pemindaian ultrasonografi berbasis AI":
% \end{minipage}

% %-------------------------------------------------	
% \noindent\fbox{%
% 	\parbox{\textwidth}{%
% \begin{enumerate}
% \item Mengidentifikasi faktor-faktor yang mempengaruhi akurasi deteksi kanker 
% payudara dengan menggunakan teknologi pemindaian ultrasonografi berbasis 
% AI.
% \item Menilai efektivitas teknologi pemindaian ultrasonografi berbasis AI dalam 
% meningkatkan akurasi deteksi kanker payudara.
% \item Menentukan metode pemindaian ultrasonografi berbasis AI yang paling efektif dalam meningkatkan akurasi deteksi kanker payudara.
% \item Menilai keamanan dan tolerabilitas teknologi pemindaian ultrasonografi berbasis AI dalam deteksi kanker payudara.
% \item Membandingkan akurasi deteksi kanker payudara dengan teknologi pemindaian 
% ultrasonografi berbasis AI dengan metode deteksi lainnya.
% \item Menyediakan bukti ilmiah untuk menunjukkan bahwa teknologi pemindaian 
% ultrasonografi berbasis AI dapat digunakan sebagai metode deteksi kanker 
% payudara yang lebih efektif dan akurat.
% \item Meningkatkan akurasi deteksi kanker payudara dengan menggunakan teknologi 
% pemindaian ultrasonografi berbasis AI.
% \end{enumerate}
		
% 	}%
% }

%-------------------------------------------------	

% \newpage
% \vspace{5mm}
% \textbf{Contoh Tujuan Penelitian Skripsi Teknologi Informasi:}

% \begin{minipage}{0.92\textwidth}
% Berikut adalah beberapa contoh tujuan penelitian untuk penelitian dengan tema 
% "Bagaimana meningkatkan efisiensi dan keamanan sistem penyimpanan data cloud 
% melalui implementasi teknologi enkripsi kuantum?":
% \end{minipage}

% %-------------------------------------------------	
% \noindent\fbox{%
% 	\parbox{\textwidth}{%
% \begin{enumerate}
% \item Menilai efektivitas implementasi teknologi enkripsi kuantum dalam 
% meningkatkan keamanan data pada sistem penyimpanan cloud.
% \item Menentukan metode enkripsi kuantum yang paling efektif dalam meningkatkan 
% keamanan data pada sistem penyimpanan cloud.
% \item Membandingkan efisiensi enkripsi kuantum dengan metode enkripsi lainnya 
% dalam meningkatkan keamanan data pada sistem penyimpanan cloud.
% \item Menilai keamanan dan stabilitas sistem penyimpanan data cloud setelah 
% implementasi teknologi enkripsi kuantum.
% \item Menyediakan bukti ilmiah untuk menunjukkan bahwa implementasi teknologi 
% enkripsi kuantum dapat meningkatkan efisiensi dan keamanan sistem 
% penyimpanan data cloud.
% \item Mengidentifikasi potensi masalah dan hambatan dalam implementasi teknologi 
% enkripsi kuantum pada sistem penyimpanan data cloud.
% \end{enumerate}
		
% 	}%
% }

%-------------------------------------------------	

\section{Batasan Penelitian}

Adapun Batasan masalah dari tugas akhir ini adalah :

\begin{itemize}
\item Penelitian tidak membahas aspek pembuatan konten 3D, konten informasi, pengambilan gambar serta desain antarmuka dan pengalaman pengguna dari, namun hanya sebatas pengembangan aplikasi web 3D virtual map tour Fakultas Teknik Universitas Gadjah Mada.
\item Pembuatan aplikasi 3D Virtual Map Tour hanya meliputi lokasi dan informasi detail tentang berbagai fasilitas dan landmark di Fakultas Teknik Universitas Gadjah Mada.
\item Aplikasi 3D Virtual Map Tour dikembangkan dengan menggunakan teknologi berbasis web sehingga dapat diakses oleh berabagai perangkat yang memiliki browser.
\item Evaluasi dan validasi kinerja aplikasi 3D Virtual Map Tour dilakukan melalui pengujian fungsionalitas, kegunaan, dan kepuasan pengguna.
\item Tidak mempertimbangkan aspek hardware atau perangkat keras yang dibutuhkan untuk mengakses aplikasi.
\end{itemize}

% \noindent \textbf{Contoh penulisan batasan skripsi Teknik Elektro:}

% %-------------------------------------------------	
% \noindent\fbox{%
% 	\parbox{\textwidth}{%
% \begin{enumerate}
% \item Objek penelitian: Studi efisiensi dan kinerja sistem pemantauan suhu dan arus pada sistem pembangkit tenaga listrik.
% \item Metode penelitian: Penelitian eksperimental menggunakan analisis simulasi dan pengujian sistem pada skala laboratorium.
% \item Waktu dan tempat penelitian: Waktu penelitian adalah Maret-Agustus 2022 di 
% FasLab TTL.
% \item Populasi dan sampel: Populasi adalah sistem pembangkit tenaga listrik, dan sampel diambil sebanyak 3 sistem yang berbeda.
% \item Variabel: Variabel bebas adalah metode pemantauan suhu dan arus, dan variabel terikat adalah efisiensi dan kinerja
% \item Hipotesis: bahwa metode pemantauan suhu dan arus berpengaruh terhadap efisiensi dan kinerja sistem pembangkit tenaga listrik.
% \item Keterbatasan Penelitian: Keterbatasan penelitian adalah hanya melibatkan 
% pengujian sistem pada skala laboratorium dan membatasi analisis pada variabel 
% bebas dan terikat.
% \end{enumerate}
		
% 	}%
% }

% %-------------------------------------------------	
% \newpage

% \noindent \textbf{Contoh penulisan batasan skripsi Teknik Biomedis:}

% %-------------------------------------------------	
% \noindent\fbox{%
% 	\parbox{\textwidth}{%
% \begin{enumerate}
% \item Objek Penelitian: Studi efektivitas dan keamanan alat elektromedik seperti pacu jantung.
% \item Metode Penelitian: Penelitian deskriptif dan observasional menggunakan survei dan analisis data.
% \item Waktu dan Tempat Penelitian: Waktu penelitian adalah Januari-Juni 2022 di 
% Rumah Sakit Sarjito. 
% \item Populasi dan Sampel: Populasi adalah pasien yang menggunakan pacu jantung, dan sampel diambil dengan metode random sampling sebanyak 100 pasien. 
% \item Variabel: Variabel bebas nya adalah jenis alat pacu jantung, dan variabel terikatnya adalah efektivitas dan keamanan. 
% \item Hipotesis: bahwa jenis alat pacu jantung berpengaruh terhadap efektivitas dan keamanan. 
% \item Keterbatasan Penelitian: Keterbatasan penelitian adalah hanya melibatkan satu rumah sakit sebagai lokasi penelitian dan membatasi dalam analisis data hanya pada variabel bebas dan terikat.
% \end{enumerate}
		
% 	}%
% }

% %-------------------------------------------------	

% \newpage
% \noindent \textbf{Contoh penulisan batasan skripsi Teknologi Informasi:}

% %-------------------------------------------------	
% \noindent\fbox{%
% 	\parbox{\textwidth}{%
% \begin{enumerate}
% \item Objek Penelitian: Analisis perbandingan efektivitas dan efisiensi antara sistem manajemen proyek tradisional dan sistem manajemen proyek berbasis teknologi informasi. 
% \item Metode Penelitian: Penelitian kualitatif dengan menggunakan wawancara dan 
% survei terhadap para pelaku proyek di berbagai perusahaan. 
% \item Waktu dan Tempat Penelitian: Waktu penelitian adalah Januari-Juni 2022 di 
% perusahaan-perusahaan di wilayah Bantul. 
% \item Populasi dan Sampel: Populasi nya adalah perusahaan yang melakukan proyek, dan sampel diambil sebanyak 10 perusahaan yang menerapkan sistem manajemen 
% proyek tradisional dan 10 perusahaan yang menggunakan sistem manajemen 
% proyek berbasis teknologi informasi. 
% \item Variabel: Variabel bebasnya adalah sistem manajemen proyek, dan variabel 
% terikatnya adalah efektivitas dan efisiensi. 
% \item Hipotesis: bahwa sistem manajemen proyek berbasis teknologi informasi lebih efektif dan efisien dibandingkan dengan sistem manajemen proyek tradisional.
% \item Keterbatasan Penelitian: Keterbatasan penelitian adalah penelitian hanya dilakukan pada perusahaan di wilayah Bantul dan hanya melibatkan wawancara dan survei sebagai metode pengumpulan data.
% \end{enumerate}
		
% 	}%
% }

\section{Manfaat Penelitian}

Manfaat yang ingin diperoleh dari penelitian ini adalah

\begin{itemize}
	\item Meningkatkan pengalaman pengguna dalam memahami dan menjelajahi Fakultas Teknik Universitas Gadjah Mada dengan cara yang lebih interaktif dan menyenangkan.
	\item Mempermudah mahasiswa, calon mahasiswa, dan pengunjung dalam mengakses informasi tentang Fakultas Teknik Universitas Gadjah Mada, termasuk lokasi gedung, fasilitas, dan kegiatan yang ada di dalamnya.
	\item Memberikan alternatif bagi pengunjung yang tidak dapat datang secara fisik ke kampus untuk mengenal Fakultas Teknik Universitas Gadjah Mada.
	\item Sebagai sarana promosi untuk Fakultas Teknik Universitas Gadjah Mada kepada masyarakat umum baik nasional maupun internasional.
	\end{itemize}

\section{Sistematika Penulisan}

Secara garis besar laporan kerja praktik ini terdiri atas lima bab yang dijelaskan sebagai berikut :

\begin{enumerate}
	\item Bab 1 : Pendahuluan
	
	Berisi mengenai beberapa hal yaitu :
	\begin{enumerate}[label=\alph*.]
		
		\item Latar belakang
		
		Penjelasan tentang alasan mengapa topik tersebut dipilih, relevansi topik dengan bidang studi, dan isu-isu yang mendorong penelitian.
		\item Rumusan masalah
		
		Identifikasi masalah: Penjelasan tentang permasalahan atau masalah yang ingin diselesaikan dalam penelitian, serta batasan masalah yang akan diteliti.
		\item Batasan tugas akhir

		Penjelasan tentang keterbatasan dalam penelitian, baik itu dari segi wilayah, waktu, atau data yang digunakan.
		\item Manfaat penelitian
		
		Penjelasan mengenai manfaat yang diharapkan dari hasil penelitian, baik bagi akademisi, praktisi, atau masyarakat umum.
		\item Sistematika penulisan
		
		Penjelasan mengenai struktur penulisan skripsi, termasuk bab-bab yang akan ditulis dan pengaturan isi dari setiap bab.
		\end{enumerate}
		\item Bab 2 : Tinjauan Pustaka dan Dasar Teori

		Berisi mengenai beberapa hal yaitu :
		\begin{enumerate}[label=\alph*.]
			\item Tinjauan Pustaka
			
			Ringkasan dari beberapa penelitian terdahulu atau sumber-sumber lain yang berkaitan dengan topik penelitian. Tinjauan pustaka ini berfungsi untuk memberikan gambaran mengenai apa yang telah diketahui dan dipelajari sebelumnya oleh para peneliti.
			\item Dasar teori
			
			Penjelasan mengenai konsep-konsep, teori-teori, dan prinsip-prinsip yang digunakan dalam penelitian. Dasar teori ini berguna untuk menjelaskan landasan teori yang digunakan dalam penelitian serta memberikan pemahaman yang lebih baik mengenai topik yang diteliti.
			\item Analisis perbandingan metode
	
			Penjelasan mengenai analisis yang dilakukan terhadap penelitian-penelitian terdahulu yang terkait dengan topik penelitian. Analisis ini berguna untuk memperjelas kelemahan atau kekurangan dari penelitian-penelitian sebelumnya dan menjelaskan bagaimana penelitian ini akan mengatasi kelemahan atau kekurangan tersebut.
		\end{enumerate}

		\item Bab 3: Metode Tugas Akhir

		Berisi mengenai beberapa hal yaitu :
		\begin{enumerate}[label=\alph*.]
			\item Alat dan Bahan
			
			Alat dan bahan tugas akhir merujuk pada peralatan, perangkat lunak, dan bahan-bahan yang digunakan dalam penelitian atau proyek tersebut.
			\item Alur tugas akhir
			
			penjelasan mengenai prosedur atau langkah-langkah yang dilakukan dalam melakukan penelitian atau proyek tugas akhir. Alur tugas akhir dapat mencakup penjelasan tentang tahapan awal seperti perencanaan dan pengumpulan data, hingga tahapan akhir seperti analisis data dan pembuatan kesimpulan. Alur tugas akhir dapat membantu pembaca memahami secara sistematis bagaimana penelitian atau proyek tersebut dilakukan dan menghasilkan hasil akhir yang diharapkan.
			\end{enumerate}
		\item Bab 4: Hasil dan Pembahasan

		dapat berupa data, gambar, tabel, atau grafik yang menjelaskan temuan penelitian. Selain itu, pada bab ini juga dapat dijelaskan bagaimana hasil tersebut berkaitan dengan pertanyaan penelitian atau tujuan proyek, dan dapat diberikan interpretasi atau penjelasan mengenai hasil yang diperoleh.
	
		\item Bab 5: Kesimpulan dan Saran

		Berisi Mengenai kesimpulan dari tugas akhir dan saran pengembangan kedepannya.
	\end{enumerate}


